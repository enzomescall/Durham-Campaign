% Options for packages loaded elsewhere
\PassOptionsToPackage{unicode}{hyperref}
\PassOptionsToPackage{hyphens}{url}
%
\documentclass[
]{article}
\usepackage{amsmath,amssymb}
\usepackage{iftex}
\ifPDFTeX
  \usepackage[T1]{fontenc}
  \usepackage[utf8]{inputenc}
  \usepackage{textcomp} % provide euro and other symbols
\else % if luatex or xetex
  \usepackage{unicode-math} % this also loads fontspec
  \defaultfontfeatures{Scale=MatchLowercase}
  \defaultfontfeatures[\rmfamily]{Ligatures=TeX,Scale=1}
\fi
\usepackage{lmodern}
\ifPDFTeX\else
  % xetex/luatex font selection
\fi
% Use upquote if available, for straight quotes in verbatim environments
\IfFileExists{upquote.sty}{\usepackage{upquote}}{}
\IfFileExists{microtype.sty}{% use microtype if available
  \usepackage[]{microtype}
  \UseMicrotypeSet[protrusion]{basicmath} % disable protrusion for tt fonts
}{}
\makeatletter
\@ifundefined{KOMAClassName}{% if non-KOMA class
  \IfFileExists{parskip.sty}{%
    \usepackage{parskip}
  }{% else
    \setlength{\parindent}{0pt}
    \setlength{\parskip}{6pt plus 2pt minus 1pt}}
}{% if KOMA class
  \KOMAoptions{parskip=half}}
\makeatother
\usepackage{xcolor}
\usepackage[margin=1in]{geometry}
\usepackage{color}
\usepackage{fancyvrb}
\newcommand{\VerbBar}{|}
\newcommand{\VERB}{\Verb[commandchars=\\\{\}]}
\DefineVerbatimEnvironment{Highlighting}{Verbatim}{commandchars=\\\{\}}
% Add ',fontsize=\small' for more characters per line
\usepackage{framed}
\definecolor{shadecolor}{RGB}{248,248,248}
\newenvironment{Shaded}{\begin{snugshade}}{\end{snugshade}}
\newcommand{\AlertTok}[1]{\textcolor[rgb]{0.94,0.16,0.16}{#1}}
\newcommand{\AnnotationTok}[1]{\textcolor[rgb]{0.56,0.35,0.01}{\textbf{\textit{#1}}}}
\newcommand{\AttributeTok}[1]{\textcolor[rgb]{0.13,0.29,0.53}{#1}}
\newcommand{\BaseNTok}[1]{\textcolor[rgb]{0.00,0.00,0.81}{#1}}
\newcommand{\BuiltInTok}[1]{#1}
\newcommand{\CharTok}[1]{\textcolor[rgb]{0.31,0.60,0.02}{#1}}
\newcommand{\CommentTok}[1]{\textcolor[rgb]{0.56,0.35,0.01}{\textit{#1}}}
\newcommand{\CommentVarTok}[1]{\textcolor[rgb]{0.56,0.35,0.01}{\textbf{\textit{#1}}}}
\newcommand{\ConstantTok}[1]{\textcolor[rgb]{0.56,0.35,0.01}{#1}}
\newcommand{\ControlFlowTok}[1]{\textcolor[rgb]{0.13,0.29,0.53}{\textbf{#1}}}
\newcommand{\DataTypeTok}[1]{\textcolor[rgb]{0.13,0.29,0.53}{#1}}
\newcommand{\DecValTok}[1]{\textcolor[rgb]{0.00,0.00,0.81}{#1}}
\newcommand{\DocumentationTok}[1]{\textcolor[rgb]{0.56,0.35,0.01}{\textbf{\textit{#1}}}}
\newcommand{\ErrorTok}[1]{\textcolor[rgb]{0.64,0.00,0.00}{\textbf{#1}}}
\newcommand{\ExtensionTok}[1]{#1}
\newcommand{\FloatTok}[1]{\textcolor[rgb]{0.00,0.00,0.81}{#1}}
\newcommand{\FunctionTok}[1]{\textcolor[rgb]{0.13,0.29,0.53}{\textbf{#1}}}
\newcommand{\ImportTok}[1]{#1}
\newcommand{\InformationTok}[1]{\textcolor[rgb]{0.56,0.35,0.01}{\textbf{\textit{#1}}}}
\newcommand{\KeywordTok}[1]{\textcolor[rgb]{0.13,0.29,0.53}{\textbf{#1}}}
\newcommand{\NormalTok}[1]{#1}
\newcommand{\OperatorTok}[1]{\textcolor[rgb]{0.81,0.36,0.00}{\textbf{#1}}}
\newcommand{\OtherTok}[1]{\textcolor[rgb]{0.56,0.35,0.01}{#1}}
\newcommand{\PreprocessorTok}[1]{\textcolor[rgb]{0.56,0.35,0.01}{\textit{#1}}}
\newcommand{\RegionMarkerTok}[1]{#1}
\newcommand{\SpecialCharTok}[1]{\textcolor[rgb]{0.81,0.36,0.00}{\textbf{#1}}}
\newcommand{\SpecialStringTok}[1]{\textcolor[rgb]{0.31,0.60,0.02}{#1}}
\newcommand{\StringTok}[1]{\textcolor[rgb]{0.31,0.60,0.02}{#1}}
\newcommand{\VariableTok}[1]{\textcolor[rgb]{0.00,0.00,0.00}{#1}}
\newcommand{\VerbatimStringTok}[1]{\textcolor[rgb]{0.31,0.60,0.02}{#1}}
\newcommand{\WarningTok}[1]{\textcolor[rgb]{0.56,0.35,0.01}{\textbf{\textit{#1}}}}
\usepackage{graphicx}
\makeatletter
\def\maxwidth{\ifdim\Gin@nat@width>\linewidth\linewidth\else\Gin@nat@width\fi}
\def\maxheight{\ifdim\Gin@nat@height>\textheight\textheight\else\Gin@nat@height\fi}
\makeatother
% Scale images if necessary, so that they will not overflow the page
% margins by default, and it is still possible to overwrite the defaults
% using explicit options in \includegraphics[width, height, ...]{}
\setkeys{Gin}{width=\maxwidth,height=\maxheight,keepaspectratio}
% Set default figure placement to htbp
\makeatletter
\def\fps@figure{htbp}
\makeatother
\setlength{\emergencystretch}{3em} % prevent overfull lines
\providecommand{\tightlist}{%
  \setlength{\itemsep}{0pt}\setlength{\parskip}{0pt}}
\setcounter{secnumdepth}{-\maxdimen} % remove section numbering
\ifLuaTeX
  \usepackage{selnolig}  % disable illegal ligatures
\fi
\IfFileExists{bookmark.sty}{\usepackage{bookmark}}{\usepackage{hyperref}}
\IfFileExists{xurl.sty}{\usepackage{xurl}}{} % add URL line breaks if available
\urlstyle{same}
\hypersetup{
  pdftitle={EDA},
  pdfauthor={Enzo Moraes Mescall},
  hidelinks,
  pdfcreator={LaTeX via pandoc}}

\title{EDA}
\author{Enzo Moraes Mescall}
\date{2024-02-26}

\begin{document}
\maketitle

\begin{Shaded}
\begin{Highlighting}[]
\CommentTok{\# Importing data}
\NormalTok{path }\OtherTok{=} \StringTok{"./data/SC {-} Endorsment Data Analysis {-} "}
\NormalTok{precinct\_df }\OtherTok{=} \FunctionTok{read\_csv}\NormalTok{(}\FunctionTok{paste0}\NormalTok{(path, }\StringTok{"precinct.csv"}\NormalTok{))}
\NormalTok{candidates\_df }\OtherTok{=} \FunctionTok{read\_csv}\NormalTok{(}\FunctionTok{paste0}\NormalTok{(path, }\StringTok{"candidates.csv"}\NormalTok{))}

\NormalTok{aj\_v\_leo }\OtherTok{=} \FunctionTok{read\_csv}\NormalTok{(}\FunctionTok{paste0}\NormalTok{(path, }\StringTok{"AJ vs. Leo.csv"}\NormalTok{))}
\NormalTok{baker\_v\_caballero }\OtherTok{=} \FunctionTok{read\_csv}\NormalTok{(}\FunctionTok{paste0}\NormalTok{(path, }\StringTok{"Baker vs. Caballero vs. Rist vs. et. al.csv"}\NormalTok{))}
\NormalTok{johnson\_v\_reece }\OtherTok{=} \FunctionTok{read\_csv}\NormalTok{(}\FunctionTok{paste0}\NormalTok{(path, }\StringTok{"Johnson vs. Reece vs. Caballero vs. et. al.csv"}\NormalTok{))}
\NormalTok{murdock\_v\_freelon }\OtherTok{=} \FunctionTok{read\_csv}\NormalTok{(}\FunctionTok{paste0}\NormalTok{(path, }\StringTok{"Murdock vs. Freelon.csv"}\NormalTok{))}
\NormalTok{satana\_v\_echols }\OtherTok{=} \FunctionTok{read\_csv}\NormalTok{(}\FunctionTok{paste0}\NormalTok{(path, }\StringTok{"Satana vs. Echols (County Wide).csv"}\NormalTok{))}
\NormalTok{schewel\_v\_ali }\OtherTok{=} \FunctionTok{read\_csv}\NormalTok{(}\FunctionTok{paste0}\NormalTok{(path, }\StringTok{"Schewel vs. Ali.csv"}\NormalTok{))}
\end{Highlighting}
\end{Shaded}

\hypertarget{methodology}{%
\subsection{Methodology}\label{methodology}}

Build a table which looks like this:

Result \textbar{} Precinct \textbar{} {[}Candidate 1 info {]} \textbar{}
{[}Candidate 2 info{]} \textbar{} {[}Population info{]}

Where results is the gap between the to candidates. In a multi-candidate
election we can artificially turn it into a series of head-to-head
races. We then want to regress on result as the dependent variable and
identify the independent variables that are most predictive of the
result.

\begin{Shaded}
\begin{Highlighting}[]
\CommentTok{\# Buliding the master table}

\CommentTok{\# Demographic percentages of possible voters in the precinct}
\NormalTok{precinct\_pct }\OtherTok{=} \FunctionTok{data.frame}\NormalTok{(}\FunctionTok{lapply}\NormalTok{(precinct\_df }\SpecialCharTok{\%\textgreater{}\%} \FunctionTok{select}\NormalTok{(}\SpecialCharTok{{-}}\NormalTok{precinct\_id), }\ControlFlowTok{function}\NormalTok{(x) x}\SpecialCharTok{/}\NormalTok{precinct\_df}\SpecialCharTok{$}\NormalTok{Total)) }\SpecialCharTok{\%\textgreater{}\%}
  \FunctionTok{mutate}\NormalTok{(}\AttributeTok{row =} \FunctionTok{row\_number}\NormalTok{()) }\SpecialCharTok{\%\textgreater{}\%}
  \FunctionTok{right\_join}\NormalTok{(precinct\_df }\SpecialCharTok{\%\textgreater{}\%} \FunctionTok{select}\NormalTok{(precinct\_id) }\SpecialCharTok{\%\textgreater{}\%} \FunctionTok{mutate}\NormalTok{(}\AttributeTok{row =} \FunctionTok{row\_number}\NormalTok{()), }\AttributeTok{by =} \StringTok{"row"}\NormalTok{) }\SpecialCharTok{\%\textgreater{}\%}
  \FunctionTok{select}\NormalTok{(}\SpecialCharTok{{-}}\NormalTok{row, }\SpecialCharTok{{-}}\NormalTok{Total) }\SpecialCharTok{\%\textgreater{}\%}
  \FunctionTok{select}\NormalTok{(precinct\_id, }\FunctionTok{everything}\NormalTok{()) }\SpecialCharTok{\%\textgreater{}\%}
  \FunctionTok{rename}\NormalTok{(}\StringTok{"Precinct"} \OtherTok{=} \StringTok{"precinct\_id"}\NormalTok{) }
\end{Highlighting}
\end{Shaded}

\begin{Shaded}
\begin{Highlighting}[]
\NormalTok{pct\_to\_dec }\OtherTok{=} \ControlFlowTok{function}\NormalTok{(string) \{}
\NormalTok{  string }\OtherTok{=} \FunctionTok{gsub}\NormalTok{(}\StringTok{"\%"}\NormalTok{, }\StringTok{""}\NormalTok{, string)}
\NormalTok{  string }\OtherTok{=} \FunctionTok{as.numeric}\NormalTok{(string)}
\NormalTok{  string }\OtherTok{=}\NormalTok{ string}\SpecialCharTok{/}\DecValTok{100}
  \FunctionTok{return}\NormalTok{(string)}
\NormalTok{\}}


\CommentTok{\# Creating a function to clean up race dataframes}
\NormalTok{head\_to\_head }\OtherTok{=} \ControlFlowTok{function}\NormalTok{(candidate\_df, candidate\_names)\{}
  \CommentTok{\# Initialize hth\_df with Precinct, c1\_name, c2\_name, result, c1\_pct, c2\_pct}
\NormalTok{  hth\_df }\OtherTok{=} \FunctionTok{data.frame}\NormalTok{(}\AttributeTok{Precinct =} \FunctionTok{character}\NormalTok{(),}
                      \AttributeTok{c1\_name =} \FunctionTok{character}\NormalTok{(),}
                      \AttributeTok{c2\_name =} \FunctionTok{character}\NormalTok{(),}
                      \AttributeTok{result =} \FunctionTok{numeric}\NormalTok{(),}
                      \AttributeTok{c1\_pct =} \FunctionTok{numeric}\NormalTok{(),}
                      \AttributeTok{c2\_pct =} \FunctionTok{numeric}\NormalTok{())}
  
  \CommentTok{\# Create a matrix of all combination of candidate\_names}
\NormalTok{  candidate\_combinations }\OtherTok{=} \FunctionTok{expand.grid}\NormalTok{(candidate\_names, candidate\_names) }\SpecialCharTok{\%\textgreater{}\%}
    \FunctionTok{filter}\NormalTok{(Var1 }\SpecialCharTok{!=}\NormalTok{ Var2) }\SpecialCharTok{\%\textgreater{}\%}
    \FunctionTok{mutate}\NormalTok{(}\AttributeTok{Var1 =} \FunctionTok{as.character}\NormalTok{(Var1), }\AttributeTok{Var2 =} \FunctionTok{as.character}\NormalTok{(Var2)) }\SpecialCharTok{\%\textgreater{}\%}
    \FunctionTok{filter}\NormalTok{(Var1 }\SpecialCharTok{\textless{}}\NormalTok{ Var2)}
  
  \CommentTok{\# Loop through the combinations of two names and create a new dataframe}
  \ControlFlowTok{for}\NormalTok{(i }\ControlFlowTok{in} \DecValTok{1}\SpecialCharTok{:}\FunctionTok{nrow}\NormalTok{(candidate\_combinations))\{}
\NormalTok{    candidate\_1 }\OtherTok{=}\NormalTok{ candidate\_combinations}\SpecialCharTok{$}\NormalTok{Var1[i]}
\NormalTok{    candidate\_2 }\OtherTok{=}\NormalTok{ candidate\_combinations}\SpecialCharTok{$}\NormalTok{Var2[i]}
    
\NormalTok{    hth }\OtherTok{=}\NormalTok{ candidate\_df }\SpecialCharTok{\%\textgreater{}\%}
      \FunctionTok{mutate}\NormalTok{(}\AttributeTok{c1\_pct =} \FunctionTok{pct\_to\_dec}\NormalTok{(}\FunctionTok{get}\NormalTok{(}\FunctionTok{paste0}\NormalTok{(}\StringTok{"\%"}\NormalTok{, candidate\_1))),}
             \AttributeTok{c2\_pct =} \FunctionTok{pct\_to\_dec}\NormalTok{(}\FunctionTok{get}\NormalTok{(}\FunctionTok{paste0}\NormalTok{(}\StringTok{"\%"}\NormalTok{, candidate\_2))),}
             \AttributeTok{result =}\NormalTok{ c1\_pct }\SpecialCharTok{{-}}\NormalTok{ c2\_pct,}
             \AttributeTok{c1\_name =}\NormalTok{ candidate\_1,}
             \AttributeTok{c2\_name =}\NormalTok{ candidate\_2) }\SpecialCharTok{\%\textgreater{}\%}
      \FunctionTok{select}\NormalTok{(Precinct, c1\_name, c2\_name, result, c1\_pct, c2\_pct)}
    
    \CommentTok{\# Append hth to the bottom of hth\_df}
\NormalTok{    hth\_df }\OtherTok{=} \FunctionTok{bind\_rows}\NormalTok{(hth\_df, hth)}
\NormalTok{  \}}
  \FunctionTok{return}\NormalTok{(hth\_df)}
\NormalTok{\}}
\end{Highlighting}
\end{Shaded}

\begin{Shaded}
\begin{Highlighting}[]
\NormalTok{aj\_v\_leo\_hth }\OtherTok{=} \FunctionTok{head\_to\_head}\NormalTok{(aj\_v\_leo, }\FunctionTok{c}\NormalTok{(}\StringTok{"AJ Williams"}\NormalTok{, }\StringTok{"Leonardo Williams"}\NormalTok{))}
\NormalTok{baker\_v\_caballero\_hth }\OtherTok{=} \FunctionTok{head\_to\_head}\NormalTok{(baker\_v\_caballero, }\FunctionTok{c}\NormalTok{(}\StringTok{"Nate Baker"}\NormalTok{, }\StringTok{"Javiera Caballero"}\NormalTok{, }\StringTok{"Carl Rist"}\NormalTok{, }\StringTok{"Khalilah Karim"}\NormalTok{, }\StringTok{"Monique Holsey{-}Hyman"}\NormalTok{, }\StringTok{"Shelia Ann Huggins"}\NormalTok{))}
\NormalTok{johnson\_v\_reece\_hth }\OtherTok{=} \FunctionTok{head\_to\_head}\NormalTok{(johnson\_v\_reece, }\FunctionTok{c}\NormalTok{(}\StringTok{"Jillian Johnson"}\NormalTok{, }\StringTok{"Charlie Reece"}\NormalTok{, }\StringTok{"Javiera Caballero"}\NormalTok{, }\StringTok{"Daniel Meier"}\NormalTok{, }\StringTok{"Jacqueline Wagstaff"}\NormalTok{))}
\NormalTok{murdock\_v\_freelon\_hth }\OtherTok{=} \FunctionTok{head\_to\_head}\NormalTok{(murdock\_v\_freelon, }\FunctionTok{c}\NormalTok{(}\StringTok{"Natalie Murdock"}\NormalTok{, }\StringTok{"Pierce Freelon"}\NormalTok{, }\StringTok{"Gray Ellis"}\NormalTok{))}
\NormalTok{satana\_v\_echols\_hth }\OtherTok{=} \FunctionTok{head\_to\_head}\NormalTok{(satana\_v\_echols, }\FunctionTok{c}\NormalTok{(}\StringTok{"Satana Deberry"}\NormalTok{, }\StringTok{"Roger Echols"}\NormalTok{, }\StringTok{"Daniel Meier"}\NormalTok{))}
\NormalTok{schewel\_v\_ali\_hth }\OtherTok{=} \FunctionTok{head\_to\_head}\NormalTok{(schewel\_v\_ali, }\FunctionTok{c}\NormalTok{(}\StringTok{"Steve Schewel"}\NormalTok{, }\StringTok{"Farad Ali"}\NormalTok{))}
\end{Highlighting}
\end{Shaded}

\begin{Shaded}
\begin{Highlighting}[]
\CommentTok{\# binding all}
\NormalTok{hth\_df }\OtherTok{=} \FunctionTok{bind\_rows}\NormalTok{(aj\_v\_leo\_hth, baker\_v\_caballero\_hth, johnson\_v\_reece\_hth, murdock\_v\_freelon\_hth, satana\_v\_echols\_hth, schewel\_v\_ali\_hth)}
\end{Highlighting}
\end{Shaded}

\begin{Shaded}
\begin{Highlighting}[]
\CommentTok{\# Joining with precinct\_pct and candidates\_df}
\NormalTok{full }\OtherTok{=}\NormalTok{ hth\_df }\SpecialCharTok{\%\textgreater{}\%}
  \FunctionTok{left\_join}\NormalTok{(candidates\_df }\SpecialCharTok{\%\textgreater{}\%} \FunctionTok{select}\NormalTok{(}\SpecialCharTok{{-}}\NormalTok{id), }\AttributeTok{by =} \FunctionTok{c}\NormalTok{(}\StringTok{"c1\_name"} \OtherTok{=} \StringTok{"name"}\NormalTok{)) }\SpecialCharTok{\%\textgreater{}\%}
  \FunctionTok{left\_join}\NormalTok{(candidates\_df }\SpecialCharTok{\%\textgreater{}\%} \FunctionTok{select}\NormalTok{(}\SpecialCharTok{{-}}\NormalTok{id, year), }\AttributeTok{by =} \FunctionTok{c}\NormalTok{(}\StringTok{"c2\_name"} \OtherTok{=} \StringTok{"name"}\NormalTok{), }\AttributeTok{suffix =} \FunctionTok{c}\NormalTok{(}\StringTok{".c1"}\NormalTok{, }\StringTok{".c2"}\NormalTok{)) }\SpecialCharTok{\%\textgreater{}\%}
  \FunctionTok{left\_join}\NormalTok{(precinct\_pct, }\AttributeTok{by =} \StringTok{"Precinct"}\NormalTok{) }
\end{Highlighting}
\end{Shaded}

\begin{verbatim}
## Warning in left_join(., candidates_df %>% select(-id), by = c(c1_name = "name")): Detected an unexpected many-to-many relationship between `x` and `y`.
## i Row 65 of `x` matches multiple rows in `y`.
## i Row 15 of `y` matches multiple rows in `x`.
## i If a many-to-many relationship is expected, set `relationship =
##   "many-to-many"` to silence this warning.
\end{verbatim}

\begin{verbatim}
## Warning in left_join(., candidates_df %>% select(-id, year), by = c(c2_name = "name"), : Detected an unexpected many-to-many relationship between `x` and `y`.
## i Row 395 of `x` matches multiple rows in `y`.
## i Row 14 of `y` matches multiple rows in `x`.
## i If a many-to-many relationship is expected, set `relationship =
##   "many-to-many"` to silence this warning.
\end{verbatim}

\hypertarget{model}{%
\subsection{Model}\label{model}}

Let \(y_1,\dots,y_n\) be independent random variables such that
\(y_i \sim Beta(\mu, \phi)\) where \(E(y) = \mu\) and unknown precision
\(\phi\). The logit beta regression model will rely on the following
parameters:

\begin{itemize}
\tightlist
\item
  \(y_i\) is the observed percentage of a head-to-head race between
  candidates \(c_1\) and \(c_2\) in precinct \(i\).
\item
  Fixed effect coefficients:

  \begin{itemize}
  \tightlist
  \item
    \(\beta_0\) is the intercept
  \item
    \(\beta_1\) for whichever endorsement we are interested in
  \item
    \(\beta_2\) for \texttt{incubent}
  \item
    \(\beta_3\) for interaction between \texttt{race} and `race.c1``
  \item
    \(\beta_4\) for interaction between \texttt{race} and `race.c2``
  \item
    \(\beta_5\) for \texttt{age.c1}
  \item
    \(\beta_6\) for \texttt{age.c2}
  \end{itemize}
\item
  Random effects: \(\(b_1(\text{Precinct}_i)\)\) is the random intercept
  for the i-th precinct
  \(\(b_2(\text{Precinct}_i) \times \text{Age}_{j[i]}\)\) is the random
  intercept for the j-th age group nested within the i-th precinct.
  Finally, \(\epsilon_{ij}\) represents the residual error for the j-th
  age group in the i-th precinct.
\end{itemize}

\begin{Shaded}
\begin{Highlighting}[]
\FunctionTok{library}\NormalTok{(lme4)}
\end{Highlighting}
\end{Shaded}

\begin{verbatim}
## Loading required package: Matrix
\end{verbatim}

\begin{verbatim}
## 
## Attaching package: 'Matrix'
\end{verbatim}

\begin{verbatim}
## The following objects are masked from 'package:tidyr':
## 
##     expand, pack, unpack
\end{verbatim}

\begin{Shaded}
\begin{Highlighting}[]
\CommentTok{\# family = list(family="beta",link="logit")}

\NormalTok{model }\OtherTok{=} \FunctionTok{lmer}\NormalTok{(result }\SpecialCharTok{\textasciitilde{}}\NormalTok{ (}\DecValTok{1}\SpecialCharTok{|}\NormalTok{Precinct) }\SpecialCharTok{+}\NormalTok{ PA.c1 }\SpecialCharTok{+}\NormalTok{ Indy.c1 }\SpecialCharTok{+}\NormalTok{ Committee.c1 }\SpecialCharTok{+}\NormalTok{ PA.c2 }\SpecialCharTok{+}\NormalTok{ Indy.c2 }\SpecialCharTok{+}\NormalTok{ Committee.c2, }\AttributeTok{data =}\NormalTok{ full)}
\end{Highlighting}
\end{Shaded}

\begin{verbatim}
## boundary (singular) fit: see help('isSingular')
\end{verbatim}

\begin{Shaded}
\begin{Highlighting}[]
\FunctionTok{summary}\NormalTok{(model)}
\end{Highlighting}
\end{Shaded}

\begin{verbatim}
## Linear mixed model fit by REML ['lmerMod']
## Formula: result ~ (1 | Precinct) + PA.c1 + Indy.c1 + Committee.c1 + PA.c2 +  
##     Indy.c2 + Committee.c2
##    Data: full
## 
## REML criterion at convergence: -1434.7
## 
## Scaled residuals: 
##     Min      1Q  Median      3Q     Max 
## -4.6474 -0.3046 -0.0154  0.3811  5.7159 
## 
## Random effects:
##  Groups   Name        Variance Std.Dev.
##  Precinct (Intercept) 0.00000  0.0000  
##  Residual             0.02158  0.1469  
## Number of obs: 1483, groups:  Precinct, 91
## 
## Fixed effects:
##               Estimate Std. Error t value
## (Intercept)  -0.050561   0.022792  -2.218
## PA.c1         0.150461   0.015522   9.693
## Indy.c1      -0.024970   0.015184  -1.645
## Committee.c1  0.005119   0.010216   0.501
## PA.c2        -0.053685   0.023307  -2.303
## Indy.c2      -0.023490   0.010376  -2.264
## Committee.c2 -0.052159   0.021942  -2.377
## 
## Correlation of Fixed Effects:
##             (Intr) PA.c1  Indy.1 Cmmt.1 PA.c2  Indy.2
## PA.c1       -0.250                                   
## Indy.c1      0.044 -0.832                            
## Committe.c1 -0.422  0.274 -0.320                     
## PA.c2       -0.857  0.268 -0.187  0.321              
## Indy.c2      0.024 -0.208  0.201 -0.036 -0.434       
## Committe.c2 -0.901  0.121 -0.083  0.396  0.899 -0.201
## optimizer (nloptwrap) convergence code: 0 (OK)
## boundary (singular) fit: see help('isSingular')
\end{verbatim}

\hypertarget{todo}{%
\subsection{Todo}\label{todo}}

\begin{itemize}
\tightlist
\item
  turn endorsements into a factor
\item
  build precinct random effect precinct table
\item
  decide on random effect between precinct and candidate
\end{itemize}

\end{document}
